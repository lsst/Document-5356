\subsubsection{The Event System}\label{sMw-ev}

The event system is a set of components that can send and receive reliable, 
asynchronous messages.  Components using the event system API subscribe to a 
\textit{topic} to send and receive messages.   A message sent to a topic will be sent
to all current subscribers on that topic.    The subscriber can then decide 
what to do with that message based on its contents.   For example, in DC2, one
message sent to a topic would cause one subscriber to the start the next stage
of a pipeline,  while another subscriber to that same topic would simply 
store the entire message in a database for later analysis. 

The event system uses libraries and brokers that implement the Java Message 
Service (JMS) protocol, a widely used, robust, open source standard. There are
a number of commercial and open source distributions of JMS available.  The 
Event System API was implemented using the open source C++ messaging library 
ActiveMQ-CPP\footnote{http://activemq.apache.org}.
The Event System API is a layer above ActiveMQ-CPP; we used 
Swig to expose this API to Python. 

The Event System API hides the configuration requirements of the ActiveMQ-CPP
library, simplifying the creation of components that send and receive
messages.  To create a message transmitter, the component can make one call 
specifying the location of the message broker, and the topic to send
subsequent messages to.   A component wishing to receive messages would 
create a message receiver specifying the message broker and the topic to 
subscribe to.    Messages send on a topic by the event transmitter would then
be received by the component that created the event receiver.   Creation of
these transmitters and receivers can be done individually, or globally to 
further simplify configuration by using policy files.  There are also
interfaces to allow out of order message retrieval and message reception
timeouts.

The Logging API was used to transform logging messages into events.  These
events were sent out on a logging topic.   We used the open source messaging
platform \code{Mule}\footnote{http://mule.mulesource.org}
to create a client that received event messages from the logging
topic and sent them to a database.  Events for pipeline coordination were also
sent to a database.  These events were then later searched to identify 
problems, and to do analysis of completed runs.
