% section 1: Introduction

\section{Introduction}

Data Challenge 2 (DC2) is the second in a series of prototypes of
the LSST Data Management System (DMS).  Through these data challenges,
we seek to identify the most challenging technical problems to
building a DMS that meets the LSST science goals.  We prototype
specific solutions to these challenges with the expectation that by
the start of the construction phase of the telescope, we will have a
well-defined plan for how to build a DMS that can perform at the
level needed by first light.  Despite the prototyping nature of the
data challenges, we are not producing throw-away code; rather, we
expect that the software we produce in the data challenges will serve as
the foundation for the DMS that will be completed during the construction
phase.  

In Data Challenge 1 (DC1), we focused on the DMS middleware design for
supporting nightly processing.  In particular, we designed and
prototyped the pipeline framework that would host the scientific
algorithms used to process the data coming from the telescope.  DC1
did not include actual implementations of scientific algorithms but
rather tunable \emph{resource consumers} that simulate the expected
computation load of the algorithms.  In DC2, we focused on replacing
these simulators with real implementations of the scientific
algorithms.  More of our software architecture was refined, including
the choice of using C++ to implement compute-intensive components and
Python for integrating components, and we built a software development
environment.  Consequently, we updated the pipeline framework into
this new architecture.  We executed our second-generation nightly pipeline
on real astronomical data from the Canada-France-Hawaii Telescope's
Legacy Survey.  
% need DC1 report reference
% need CFHTLS reference http://www.cfht.hawaii.edu/Science/CFHLS/

This report describes the results of Data Challenge 2.  We enumerate
our goals, summarize the implementation, and report on what we've
learned from it.  

\subsection{Goals of DC2}
\label{Goals}

As laid out prior to beginning implementation, the goals of Data
Challenge 2 were to:
\begin{itemize}
\item Demonstrate the use of astronomical algorithms for nightly
  processing in an LSST processing framework.
\item Further develop and demonstrate common application framework
  functionality.
\item Update the middleware implementation to enable the hosting and
  execution of C++ application classes.
\item Establish an LSST simulated object/source database for testing
  by the broader collaboration.
\item Demonstrate a database partitioning scheme.
\item Demonstrate integration of the Moving Objects Prediction
  Software (MOPS) into the LSST framework. 
\item Pilot the software development and testing process for future data
  challenges and the LSST construction phase.
\item Establish a reusable code baseline for future data challenges and
  the LSST construction phase.
\item Test the bandwidth utilization of the UDT protocol in the
  context of LSST data replication between Chile and the US.
\end{itemize}

As of this writing, all but the last goal have been addressed.  This
last goal is part of a now-expanded experiment to test data transfer
tools over long-distance networks.  The results of this work will be
discussed in a subsequent report and therefore will not be discussed
further here.  

\subsection{Metrics and Validation of DC2}

At the start of the DC2, we set down a list of deliverables that we
would need to achieve to validate that we had met our goals.  These
deliverables were:
\begin{itemize}
\item Demonstration of a nightly processing pipeline that operates
  continuously on a set of input data. This processing includes at
  least image subtraction, source detection, association, and moving
  object detection using MOPS. As stretch goals, we would also include
  image linearization (flat-fielding, bias subtraction, etc.) and WCS
  determination.
\item Production of a persistent database of source detections and
  associations as a result of the nightly processing. 
\item Execution of MOPS automatically within the nightly pipeline framework.
\item Management of all original code used in DC2 via our
  Subversion (SVN) repository.
\item Demonstration of the use of LSST framework and support classes.
\item Employment of a harness that accesses MP functionality within a
  compiled layer but which can execute application code via a Python
  interface. 
\end{itemize}

As we describe below, we have in the course of DC2 clearly met the baseline
milestones.  We did not impose formal metrics on the data quality
produced by the astronomical application code in DC2.  We have
nonetheless assessed the data quality, and comment on it in the report sections
that follow.

